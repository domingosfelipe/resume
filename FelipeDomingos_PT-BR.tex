\documentclass{domingosfelipe-resume}  

\name{Felipe Domingos}

\address{\href{https://www.linkedin.com/in/domingosfelipe/}{LinkedIn}}
\address{\href{mailto:contato@felipedomingos.com}{contato@felipedomingos.com}}
\address{\href{tel:+5551992643931}{+55 51 99264-3931}}

\begin{document}
    \resume{Resumo}{Engenheiro de software back-end com 17 anos de experiência, atuando com Java (versões 8 a 21) e Spring Framework. Sólida vivência em arquitetura de microsserviços, integração de sistemas e otimização de desempenho, com profundo conhecimento em bancos de dados SQL e NoSQL. Focado em entregar soluções robustas e escaláveis, possuo histórico de resolução de desafios complexos e liderança técnica na implementação de projetos.}
    
    \begin{rSection}{Experiências Profissionais}
    \begin{rExperienceSubsection}{CashGO}{Engenheiro de Software}{2023}{Presente}
        \item Migração de soluções back-end de Java 17 para Java 21 com Spring 3.x, garantindo compatibilidade com versões recentes e aprimorando a performance das aplicações.
        \item Análise e otimização de desempenho de microsserviços, resultando em aplicações mais eficientes e escaláveis.
        \item Monitoramento ativo para identificação de problemas críticos, compartilhando impactos e possíveis soluções com o time em reuniões diárias, facilitando a resolução ágil de incidentes.
        \item Configuração e gerenciamento de repositórios e pipelines CI/CD no GitHub, agilizando o ciclo de desenvolvimento.
        \item \textbf{Principais projetos:} Implementação de solução back-end para gestão de proprietários de imóveis, integrando dados de sistemas imobiliários de diversas regiões do Brasil. Participação ativa na implementação de solução de OCR de documentos com uso de Inteligência Artificial, automatizando a extração de informações de documentos.
    \end{rExperienceSubsection}
    \begin{rExperienceSubsection}{CWI Software}{Desenvolvedor back-end}{2021}{2023}
        \item Atuação em equipe multidisciplinar no desenvolvimento de projetos para os clientes \textbf{Sumicity} e \textbf{VIP Telecom}.
        \item Desenvolvimento e manutenção de microsserviços para sistemas de vendas e provisionamento de modems, garantindo escalabilidade e confiabilidade.
        \item Análise de integração entre APIs e otimização de desempenho, melhorando a comunicação entre serviços.
        \item Configuração de repositórios e pipelines CI/CD no GitLab, acelerando as entregas de software.
        \item \textbf{Principal Projeto:} Liderança na implementação da solução back-end de provisionamento de modems, automatizando fluxos manuais e permitindo configurações remotas pelos próprios clientes.
    \end{rExperienceSubsection}
    \begin{rExperienceSubsection}{Resource IT Solutions}{Desenvolvedor back-end}{2020}{2021}
        \item Consultor de desenvolvimento de software, atuando em equipe multidisciplinar no cliente \textbf{Dell}.
        \item Desenvolvimento de microsserviços para sistemas de contratos e garantias, garantindo escalabilidade e facilidade de manutenção.
        \item Configuração de repositórios e pipelines CI/CD no GitLab, viabilizando integração e entrega contínua.
        \item \textbf{Principal Projeto:} Implementação de soluções com Spring WebFlux, reduzindo o tempo de processamento e aumentando a confiabilidade do sistema.
    \end{rExperienceSubsection}
    \begin{rExperienceSubsection}{DBserver}{Desenvolvedor back-end}{2020}{2020}
        \item Consultor de desenvolvimento de software, atuando em equipe multidisciplinar no cliente \textbf{Sicredi-CAS}.
        \item Desenvolvimento de microsserviços para sistemas de análise de disponibilidade de crédito, garantindo desempenho e confiabilidade no processamento de dados.
        \item \textbf{Principal Projeto:} Aumento da eficiência de soluções legadas por meio de refatorações e implementação de novas soluções, substituindo completamente sistemas antigos e resultando em operações mais rápidas e seguras.
    \end{rExperienceSubsection}
    \begin{rExperienceSubsection}{SRM Asset}{Desenvolvedor back-end}{2018}{2020}
        \item Atuação em equipe multidisciplinar no desenvolvimento de soluções back-end.
        \item Análise e desenvolvimento de microsserviços para plataforma de Home Banking no Brasil e Peru, assegurando alta performance e segurança da aplicação.
        \item Configuração de repositórios e pipelines CI/CD no GitLab, suportando integração contínua e entregas frequentes.
        \item \textbf{Principal Projeto:} Liderança na implementação de solução back-end para gestão do envio de documentos em abertura e manutenção de contas, automatizando fluxos manuais e aumentando a segurança da plataforma.
        \end{rExperienceSubsection}
    \end{rSection}
    
    \begin{rSection}{Formação Acadêmica}
        \begin{rEducationSubsection}{Pontifícia Universidade Católica do Rio Grande do Sul – PUCRS}{2018}    
            \item Bacharelado em Sistemas de Informação.
        \end{rEducationSubsection}
    \end{rSection}
    
    \begin{rSection}{Certificados}
        \begin{rSkillsSubsection}
            \item \href{https://www.linkedin.com/learning/certificates/3781e6e11a2fe389fcb48387012853f893c2bdbc3decd7c57790c2c850c316be?trk=share_certificate}{GitHub}
            \item \href{https://www.linkedin.com/learning/certificates/bd30b94ff1b980132e44f68668e4a1c56b9fddb10832c9c01d9d65ce7d2b9c62?trk=share_certificate}{Docker Foundations}
        \end{rSkillsSubsection}
    \end{rSection}
    
    \begin{rSection}{Habilidades}
        \begin{rSkillsSubsection}
            \item Tecnologias: Java 8+, Spring Framework, SQL, NoSQL, AWS, Mensageria, JavaScript, React.
            \item Arquitetura de microsserviços escaláveis.
            \item Design Patterns, Arquitetura de Software, Algoritmos e Estruturas de Dados.
            \item Ferramentas e DevOps: GitHub, GitLab, CI/CD, Docker.
            \item Metodologias Ágeis (Scrum, Kanban).
            \item Liderança técnica e Gestão de Equipes.
        \end{rSkillsSubsection}
    \end{rSection}
\end{document}